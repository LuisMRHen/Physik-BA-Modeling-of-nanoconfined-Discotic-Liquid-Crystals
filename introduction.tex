\chapter{Introduction}
\vspace{2cm}


In the last century liquid crystals have received much attention as they provide both insight into statistical physics, \textit{e.g.} the nature of phase transitions and have very broad applications such as liquid crystal displays (LCDs), thermometers \cite{witonsky2001liquid} or as actuators \cite{guin2018layered}. Due to the self ordering and healing of liquid crystals, there is interest in the further development of applications. Recent findings by Sentker, Zantop ,et.al \cite{sentker2018quantized} show that a cylindrical confinement of a system of discotic liquid crystal molecules will lead to a quantized formation of concentric ring-like layers, a phase non-existent in the bulk system. This highlights how by different geometries may change the physics of liquid crystals entirely.

In this thesis we study the influence of a colloidal particle on a liquid crystal system both in bulk and in confinment.
\todo{change} We aim to study a thermodynamically equilibrated system, therefore a Monte Carlo approach is well-suited to produce phase space samples. They can be easily compared to experiments even providing more insight into the specific molecular-arrangement of the system, which is usually not easily experimentally determined.
The structure of the thesis is as follows.

We start in Chapter \ref{chap:Theory} by giving a brief introduction to used most concepts used in this work. In section \ref{sec:liquidcrystals} we review the most important properties of discotic liquid crystals for this work and the different phases that they manifest. We then present the order parameters with which the phases can be identified.
Section \ref{sec:compmethods} is dedicated to the numerical methods used in this work and illustrates these in detail. A further explanation of the implementation of these is provided in chapter \ref{chap:implementation}.
In chapter \ref{chap:results} we discuss the results of the simulations. We start with the confined system and investigate the influence of colloids of different sizes on the confined system and compare these to a the system without a colloid. We study different types of preferred  alignment of the liquid crystal on the surface of the colloid.

The last chapter examines these results and provides an outlook for the future. 